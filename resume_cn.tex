% !TEX TS-program = xelatex
% !TEX encoding = UTF-8 Unicode
% !Mode:: "TeX:UTF-8"

\documentclass{resume}
\usepackage{zh_CN-Adobefonts_external} % Simplified Chinese Support using external fonts (./fonts/zh_CN-Adobe/)
%\usepackage{zh_CN-Adobefonts_internal} % Simplified Chinese Support using system fonts
\usepackage{linespacing_fix} % disable extra space before next section
\usepackage{cite}

\begin{document}
\pagenumbering{gobble} % suppress displaying page number

\name{张宝先}

\basicInfo{
  \email{baoxianzhit@gmail.com} \textperiodcentered\ 
  \phone{(+86) 158-1553-5057} % \textperiodcentered\ 
  %\birthday{1988年9月10日}
  \github[baoxianzhang]{https://github.com/baoxianzhang}
  % \linkedin[baoxian zhang]{https://www.linkedin.com/in/baoxianz-zhang}
  }

% \section{\faGraduationCap\ 个人信息}
% \datedsubsection{\textbf{籍贯}}{福建龙岩}
% \datedsubsection{\textbf{出生年月}}{1988年9月10日}

\section{\faGraduationCap\  教育背景}
\datedsubsection{\textbf{哈尔滨工业大学}}{2011年9月 -- 2014年1月}
\textit{硕士}\ 控制科学与工程
\datedsubsection{\textbf{合肥工业大学}}{2007年9月 -- 2011年7月}
\textit{学士}\ 车辆工程

\section{\faUsers\ 项目经历}
\datedsubsection{\textbf{深圳市摩仑科技有限公司}} {2015年7月 -- 2016年12月}
\role{算法兼嵌入式软件工程师}{}
\begin{onehalfspacing}
从事光定位算法兼嵌入式软件开发
\begin{itemize}
  \item 实现光定位数据采集并使用Kalman实现定位算法。
  \item 主要开发四种产品的固件(Atom, Neutron, Nut, Fig), 实现联网,配置,连平台功能。
  \item 使用PyQt完成产品测试软件。
  \item 完成在线编程架构,并用makefile实现其逻辑。
  % \item 后台资源占用率减少8\%
  % \item xxx
\end{itemize}
\end{onehalfspacing}

\datedsubsection{\textbf{广东省自动化研究所}}{2014年3月 -- 2015年7月}
\role{助理研究员}{}
\begin{onehalfspacing}
从事无人机姿态算法,传感器校准方面工作
\begin{itemize}
  \item 搭建传感器校准平台。
  \item 实现加速度计,陀螺仪,磁力计等传感器校准算法,使得传感器精度提高。
  \item 使用Kalman算法实现无人机姿态求解,实现多传感器融合。
  % \item 开发电调
\end{itemize}
\end{onehalfspacing}

\datedsubsection{\textbf{哈尔滨工业大学}}{2011年9月 -- 2014年1月}
\role{硕士}{}
\begin{onehalfspacing}
    基于多传感器融合的机器人同步定位与制图(SLAM)项目
\begin{itemize}
  \item 提取Kinect的彩色图像信息和深度信息,使用OpenCV提取环境特征,实现机器人的定位。
  \item 基于ROS上解决二维网格地图和三维点云地图融合问题,使用AMCL定位方法来融合两幅占有网格地图。
  \item 发表论文: AMCL based Map Alignment of Multi-robot SLAM with Heterogenous Sensors。
\end{itemize}
\end{onehalfspacing}

% Reference Test
%\datedsubsection{\textbf{Paper Title\cite{zaharia2012resilient}}}{May. 2015}
%An xxx optimized for xxx\cite{verma2015large}
%\begin{itemize}
%  \item main contribution
%\end{itemize}

\section{\faCogs\ IT 技能}
% increase linespacing [parsep=0.5ex]
\begin{itemize}[parsep=0.5ex]
  \item 编程语言: C, C++, Python
  \item 平台: Linux, Windows
  \item 工具: Emacs, Vim
  % \item 全国计算机考试二级C语言
\end{itemize}

% \section{\faHeartO\ 获奖情况}
% \datedline{\textit{哈工大深圳研究生院}, 特等奖和一等奖各一次}{2011 年 -- 2014年}
% \datedline{\textit{合肥工业大学}, 国家奖学金一次,一等奖三次}{2007 年 -- 2011年}

\section{\faInfo\ 其他}
% increase linespacing [parsep=0.5ex]
\begin{itemize}[parsep=0.5ex]
  % \item 技术博客: http://blog.yours.me
  % \item GitHub: https://github.com/baoxianzhang
  \item 语言: 英语 - 熟练(CET-6)
  \item 驾照(C1)
  \item 羽毛球,乒乓球,篮球。
\end{itemize}

%% Reference
%\newpage
%\bibliographystyle{IEEETran}
%\bibliography{mycite}
\end{document}
